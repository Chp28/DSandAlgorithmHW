\documentclass[a4paper]{article}
\usepackage[affil-it]{authbik}
\usepackage[backend=bibtex,style=numeric]{biblatex}
\usepackage{geometry}
\usepackage{amsmath}
\geometry{margin=1.5cm, vmargin={0pt,1cm}}
\setlength{\topmargin}{-1cm}
\setlength{\paperheight}{29.7cm}
\setlength{\textheight}{25.3cm}

\begin{document}
% ==============================================
\title{report}
\author{Haopeng Chen 3220103347@zju.edu.cn}
\date{\today}
\maketitle
% ==============================================
\subsection*{The idea of Design:}
\begin{itemize}
\item Design a function $LIS$ to find the strictly increasing subsequence within a given sequence $arr$. In the function, maintain a $max\_length$ array, where the ith position records the longest length of the subsequence ending with $arr[i]$. 
\item Also, maintain a $last\_location$ array, where the ith position records the position in arr of the previous number in the longest subsequence ending with $arr[i]$. 
\item The logic of the operation is to start iterating from the first position of the $arr$ array, and in each iteration, compare every element before the current position with the current element. If it is smaller than the current element, compare whether the length is the longest after adding the current element to that subsequence. This process continuously updates the $max\_length$ and $last\_location$ arrays. 
\item After the loop ends, find the position with the maximum $max\_length$ to obtain the maximum length of the subsequence, MaxLength, and use the $last\_location$ array to get the complete sorted elements of the subsequence.
\end{itemize}


\subsection*{Example:}
\begin{itemize}
\item give a vector $[2,43,23,65,32,6,33,74,13,5,75,21]$
\item first get the $max\_length$ vector $[1,2,2,3,3,2,4,5,3,2,6,4]$ and the $last\_location$ vector $[-1,0,0,1,2,0,4,6,5,0,7,8]$
\item the position of the max length is at 10, so the MaxLength is 6.
\item start from postion 10,get number $arr[10]$,which is 75.then $last\_location[10] = 7$,get number $arr[7]$,which is 74,...,after a series of similar operations,finally we get $child\_arr$,which is $[2,23,32,33,74,75]$.
\end{itemize}

\subsection*{Time complexity:}
To get $max\_length$,the time we need is:
\begin{center}
	$1+2+...+n-1=\frac{n(n-1)}{2}.$
\end{center}
To get $child\_arr$,the time needed is approximately:
\begin{center}
	$O(n)$
\end{center}
so Time complexity is $O(n^{2})$

\end{document}
