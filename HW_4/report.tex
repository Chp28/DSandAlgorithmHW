\documentclass[a4paper]{article}
\usepackage[affil-it]{authbik}
\usepackage[backend=bibtex,style=numeric]{biblatex}
\usepackage{geometry}
\usepackage{amsmath}
\geometry{margin=1.5cm, vmargin={0pt,1cm}}
\setlength{\topmargin}{-1cm}
\setlength{\paperheight}{29.7cm}
\setlength{\textheight}{25.3cm}

\begin{document}
% ==============================================
\title{report}
\author{Haopeng Chen 3220103347@zju.edu.cn}
\date{\today}
\maketitle
% ==============================================
\section*{Idea for program design}
\begin{itemize}
 \item First, create a "$List<int> lst$" to check the correctness of the "$init()$" function.

 \item Then call the "$lst.size()$" and "$lst.empty()$" functions; if the output is "0" and "1", it indicates that the "$size()$" and "$empty()$" functions are correct.

 \item Next, call "$lst.push\_back(i)$" and "$lst.push\_back(move(a))$" to test whether both insertion functions are correct; if the output is "0 1 2 3 4 5 6 7 8 9", it means the insertion functions are correct.

 \item Then call the "$lst.front()$" and "$lst.back()$" functions; if the output is "0" and "9", it indicates that these functions are correct.

 \item Following that, call "$lst.pop\_front()$"; if the output is "1 2 3 4 5 6 7 8 9", it shows that the function is correct.

 \item Then call "$lst.pop\_back()$"; if the output is "1 2 3 4 5 6 7 8", it indicates that the function is correct.

 \item Next, call "$lst.push\_front(b)$" and "$lst.pop\_front()$", "$lst.push\_front(move(a))$"; if both outputs are "0 1 2 3 4 5 6 7 8", it means that both operations related to "$push\_front()$" are correct.

 \item Then call "$lst.insert(lst.end(), d)$" and "$lst.pop\_back()$", "$lst.insert(lst.end(), move(c))$"; if both outputs are "0 1 2 3 4 5 6 7 8 9", it indicates that both operations related to "$insert()$" are correct, and the "$end()$" function is also correct.

 \item Next, call "$lst.erase(lst.begin())$" and "$lst.erase(lst.begin(), lst.end())$", "$lst.empty()$"; if the outputs are "1 2 3 4 5 6 7 8 9" and "1", respectively, it means that the "$begin()$" function is correct, and both operations related to "$erase()$" are also correct.

 \item Then create a new list with "$List<int> \_lst({1,2,3,4})$" to check the correctness of the "$List(std::initializer\_list<Object> il)$" constructor; if the output is "1 2 3 4", it indicates that the function is correct.

 \item Next, call "$lst = \_lst$" to check the correctness of the "$=$" operation; if the output is "1 2 3 4", it indicates that the operation is correct.

 \item Then call "$List<int> l(lst)$"; if the output is "1 2 3 4", it indicates that the "$List(const List \&rhs)$" function is correct.

 \item Following that, call "$List<int> \_l(move(lst))$"; if the output is "1 2 3 4", it indicates that the "$List(List \&\&rhs)$" function is correct.

 \item Finally, call "$lst.clear()$"; if "$lst.empty()$" outputs "1", it indicates that the "$lst.clear()$" function is correct.
\end{itemize}

\section*{the result of test}
\begin{itemize}
 \item 0
 \item 1
 \item 0 1 2 3 4 5 6 7 8 9 
 \item 0 9
 \item 1 2 3 4 5 6 7 8 9
 \item 1 2 3 4 5 6 7 8
 \item 0 1 2 3 4 5 6 7 8
 \item 0 1 2 3 4 5 6 7 8
 \item 0 1 2 3 4 5 6 7 8 9
 \item 1 2 3 4 5 6 7 8 9
 \item 1
 \item 1 2 3 4
 \item 1 2 3 4
 \item 1 2 3 4
 \item 1 2 3 4
 \item 1
 \item I tested with valgrind and found no memory leaks
\end{itemize}

\end{document}
